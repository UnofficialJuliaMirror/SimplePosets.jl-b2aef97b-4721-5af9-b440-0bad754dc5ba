\documentclass[12pt]{amsart}
\usepackage[margin=1.25in]{geometry}
\usepackage{graphicx}
\usepackage{txfonts}
\pagestyle{plain}

\begin{document}

\begin{center}
  \textbf{Adding/Deleting Elements/Relations To/From a Poset}
\end{center}

Let $P=(X,\le)$ be a partially order set (or poset). Here $X$ is a set
of \emph{elements} and $\le$ is a reflexive, antisymmetric, transitive
\emph{relation} on $X$.

We wish to consider the following four operations: (1)~add an element
to a poset, (2)~delete an element from a poset, (3)~add a relation
to a poset, and (4)~delete a relation from a poset. 

\section*{Adding/Deleting Elements}

Operations on elements are relatively simple to describe. Let
$P=(X,\le)$ be a poset. 

If $a\notin X$, then to add the element $a$ to $P$ results in a new
poset that includes $a$ in which $a$ is related only to
itself. Formally, let $P'= P+a$ be the poset $P'=(X',\le')$ where we
have the following:
\begin{enumerate}
\item $X' = X \cup \{a\}$.
\item $\mathord{\le'} = \mathord{\le} \cup \{(a,a)\}$. That is, 
  \begin{itemize}
  \item $\forall x,y \in X, x\le'y\iff x\le y$,
  \item $\forall x \in X$, $x\not\le'a$ and $a \not\le' x$, and
  \item $a\le' a$.
  \end{itemize}
\end{enumerate}

Element deletion is also easy to describe. Deleting an element $a$
from $P$ deletes $a$ from the set $X$ and all remaining elements have
the same relations they had before. Formally, for $a \in X$, let
$P'=P-a$ be the poset $P'=(X',\le')$ where we have the following:
\begin{enumerate}
\item $X' = X - \{a\}$.
\item $\forall x,y \in X', x \le' y \iff x\le y$.
\end{enumerate}

\end{document}
