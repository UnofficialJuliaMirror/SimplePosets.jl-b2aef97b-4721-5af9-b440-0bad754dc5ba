\documentclass[12pt]{amsart}
\usepackage[margin=1.5in]{geometry}
%\usepackage{graphicx}
\usepackage{txfonts}
\pagestyle{plain}
\nofiles
\begin{document}

\begin{center}
  \textbf{Adding/Deleting Elements/Relations To/From a Poset}
\end{center}

Let $P=(X,\le)$ be a partially order set (or poset). Here $X$ is a set
of \emph{elements} and $\le$ is a reflexive, antisymmetric, transitive
\emph{relation} on $X$.

We wish to consider the following four operations: (1)~add an element
to a poset, (2)~delete an element from a poset, (3)~add a relation
to a poset, and (4)~delete a relation from a poset. 

\section*{Adding/Deleting Elements}

Operations on elements are relatively simple to describe. Let
$P=(X,\le)$ be a poset. 

If $a\notin X$, then to add the element $a$ to $P$ results in a new
poset that includes $a$ in which $a$ is related only to
itself. Formally, let $P'= P+a$ be the poset $P'=(X',\le')$ where we
have the following:
\begin{enumerate}
\item $X' = X \cup \{a\}$.
\item $\mathord{\le'} = \mathord{\le} \cup \{(a,a)\}$. That is, 
  \begin{itemize}
  \item $\forall x,y \in X, x\le'y\iff x\le y$,
  \item $\forall x \in X,\ x\not\le'a$ and $a \not\le' x$, and
  \item $a\le' a$.
  \end{itemize}
\end{enumerate}

Element deletion is also easy to describe. Deleting an element $a$
from $P$ deletes $a$ from the set $X$ and all remaining elements have
the same relations they had before. Formally, for $a \in X$, let
$P'=P-a$ be the poset $P'=(X',\le')$ where we have the following:
\begin{enumerate}
\item $X' = X - \{a\}$.
\item $\forall x,y \in X',\ x \le' y \iff x\le y$.
\end{enumerate}

Note that element addition and deletion operations need not
commute. While it is true that $(P+a)-a=P$, in general we have
$(P-a)+a\not=P$. 


\section*{Adding/Deleting Relations}

Adding a relation to a poset requires us to include additional
relations implied by transitivity. Let $P=(X,\le)$ be a poset
containing incomparable elements $a$ and $b$.

We define $P+(a<b)$ to be the poset $P' = (X',\le')$ in which we have
the following:
\begin{itemize}
\item $X' = X$.
\item $\forall x,y \in X', \ x \le' y \iff (x \le y) \text{ or } 
  (x \le a \text{ and } b \le y)$.
\end{itemize}
Stated differently, $\le'$ is the minimal superset of $\le$ that
includes the pair $(a,b)$ and that is reflexive, antisymmetric, and
transitive. 

There does not appear to be ``best'' way to define relation
deletion. Suppose $P=(X,\le)$ is a poset in which $a<b$; we want to
define $P' = P-(a<b)$.  For example, suppose $P=([3],\le)$ is the
total order $1<2<3$. How shall we define $P-(1<3)$? Since we delete
$(1,3)$ from the relation, we cannot have both $1<2$ and $2<3$, so one
of those must be deleted as well.  This leads to two possible choices
for $\le'$ are these:
\begin{itemize}
\item $\mathord{\le'} = \{(1,1),(2,2),(3,3), (1,2) \}$ and
\item $\mathord{\le'} = \{(1,1),(2,2),(3,3), (2,3) \}$.
\end{itemize}
There's no reasonable way to choose between these alternatives. Both
are derived from $\le$ with a minimum number of changes. So we take
another approach by deleting both $1<2$ and $2<3$. This results in the
antichain on $[3]$. 

More generally, when we delete $a<b$ from $P$ we need to delete other
relations. In particular, if there is an $x$ with $a<x<b$, we cannot
keep both $a<x$ and $x<b$. Our solution is to delete \emph{both}. 

Thus we define $P-(a<b)$ to be the poset $P' = (X',\le')$ in which 
$X'=X$ and
\[
\mathord{\le'} = \mathord{\le} 
- \bigl\{ (a,b) \bigr\}
- \bigl\{(a,x),(x,b) : a<x<b \bigr\} .
\]

\noindent\textbf{Claim}. \emph{$P'$ is a poset.}

\begin{proof}
  We need to check that $\le'$ is reflexive, antisymmetric, and
  transitive. 

  Since we have not deleted any relation of the form $(x,x)$ from
  $\le$, it follows that $\le'$ is reflexive.

  Since $\mathord{\le'} \subset \mathord{\le}$ it follows that 
  \[
  (x \le' y \text{ and } y \le' x) \Rightarrow
  (x \le y \text{ and } y \le x) \Rightarrow x=y .
  \]

  Finally, we must show that $\le'$ is transitive. Suppose $x <' y <'
  z$ but we do not have $x <' z$. This means that $(x,z)$ was a
  relation deleted from $\le$ and so we have one of the following:
  \begin{enumerate}
  \item $(x,z) = (a,b)$,
  \item $a=x < z < b$, or
  \item $a<x<z=b$.
  \end{enumerate}
  
  Case (1) cannot hold because then we have $x=a<y<b=z$ in which case
  neither $x<'y$ nor $y<'z$ contradicting the supposition that
  $x<'y<'z$. 

  In case (2) we have that $a=x < y < z < b$ contradicting $x <' y$,
  and a similar contradiction holds in case~(3). 
\end{proof}

\end{document}
